\documentclass[main.tex]{subfiles}
\begin{document}
	\section{Run}
	\subsection{Knowledge}
	\subsubsection{Knowledge}
	To start knowledge you need to run the following command.
	\begin{itemize}
\item roslaunch knowledge knowledge.launch
\end{itemize}
If you want to manually interact with the knowledge base you can run 
\begin{itemize}
\item \begin{verbatim}rosrun rosprolog rosprolog_commandline.py \end{verbatim}
\end{itemize}
to start a SWI-Prolog interface.
	
	\subsubsection{NLP}
	
	To start the hard commands recognition run the following.
	
	\begin{itemize}
		\item \begin{verbatim}roslaunch suturo_spechparser hard_commands_only.launch \end{verbatim}
	\end{itemize}
	
	\subsection{Perception}

	To start perception you first need to start the pipelines:
	\begin{itemize}
\item \begin{verbatim} roscd rs_perception \end{verbatim} 
\item \begin{verbatim}
roslaunch rs_perception hsrb_perception.launch
\end{verbatim}
\end{itemize}	

After that start the action clients:
\begin{itemize}
\item rosrun actionserver ExtractObjectInfoClient
\item rosrun actionserver ExtractPlaneInfoClient
\end{itemize}

To Limit the perception of objects to a specific region you can use:
\begin{itemize}
\item \begin{verbatim} set_region:=region name \end{verbatim} 
\end{itemize}
Region names:
\begin{itemize}
\item \begin{verbatim} bed_table- table_clone \end{verbatim} 
\item \begin{verbatim} table \end{verbatim} 
\item \begin{verbatim} hcr_shefl_2_0 \end{verbatim} 
\item \begin{verbatim} hcr_shefl_2_1 \end{verbatim}
\item \begin{verbatim} hcr_shefl_2_2 \end{verbatim}
\item \begin{verbatim} bookshelf_clone_0_0 \end{verbatim} 
\item \begin{verbatim} bookshelf_clone_0_1 \end{verbatim} 
\item \begin{verbatim} bookshelf_clone_0_2 \end{verbatim} 
\item \begin{verbatim} bookshelf_0 \end{verbatim} 
\item \begin{verbatim} bookshelf_1 \end{verbatim} 
\item \begin{verbatim} bookshelf_2 \end{verbatim} 
\item \begin{verbatim} robocup_default \end{verbatim} 
\end{itemize}

The result topics are the following.
\begin{itemize}
\item \begin{verbatim}
perception_actionserver/result
\end{verbatim}
\item \begin{verbatim}
perception_actionserver_plane/result
\end{verbatim}
\end{itemize}

The action client is not necessary for the HSR demo.
	
	\subsection{Manipulation}
	To start the four Manipulation servers run the following command:
	\begin{itemize}
		\item roslaunch suturo\_manipulation\_launch start\_manipulation\_servers.launch
	\end{itemize}
	If the start succeded every server will return a message that it successfully started in the terminal.
	\subsubsection{Manipulation}
	
	\subsubsection{Navigation}
	Run the following comman start navigation:
	\begin{itemize}
		\item roslaunch suturo\_navigation\_launch start\_suturo\_navigation.launch
	\end{itemize}
	
	\subsection{Planning}
	Before you start planning the simulation environment, knowledge, perception and manipulation must be launched.\\
	The first step to launching planning is the following command.
	\begin{itemize}
\item \begin{verbatim} roslisp_repl \end{verbatim} 
\end{itemize}
	\subsubsection{Groceries}
	To start the grocery plan you now need to load the package \begin{verbatim}
	grocery_demo
	\end{verbatim}
	 using the following commands:
	\begin{itemize}
\item (swank:operate-on-system-for-emacs "grocery-demo" (quote load-op))
\item (grocery::init-interfaces)
\end{itemize}
This step takes some time. After that the task can be started:
\begin{itemize}
\item (grocery::execute-grocery)
\end{itemize}

	\subsubsection{Cleanup}
	To start the cleanup plan you now need to load the package \begin{verbatim}
	cleanup_demo
	\end{verbatim}
	 using the following commands:
	\begin{itemize}
\item (swank:operate-on-system-for-emacs "cleanup-demo" (quote load-op))
\item (clean::init-interfaces)
\end{itemize}
This step takes some time. After that the task can be started:
\begin{itemize}
\item (clean::execute-cleanup)
\end{itemize}

\end{document}
