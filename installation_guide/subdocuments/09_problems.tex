\documentclass[main.tex]{subfiles}
\begin{document}
	\section{Known problems}
	\label{problems}
	\subsection{Workspaces}
	\subsubsection{Manipulation and Knowledge}
	It is known, that some messages can cause problems between Manipulation and Knowledge. Because of this we installed Manipulation in a different workspace.
	
	\subsubsection{Planning}
	If you want to install Planning in the same workspace as the other parts, the hsr\_description package should cause problems. Because of this issue we use a separate workspace for Planning.
	

	\subsection{Knowledge}
	\subsubsection{Gradle}
	Sometimes while building Knowledge, gradle causes problems.
	If that happens it usually can be fixed by deleting the .gradle folder in your home folder.
	
	\subsubsection{Ros-mongodb\_log}
	Running the \\
	\begin{lstlisting}
rosdep install
\end{lstlisting}
command in a workspace with Knowledge can cause problems with the package ros-mongodb\_log. To solve this you can use the option -r of the \\
\begin{lstlisting}
rosdep install
\end{lstlisting}
command.\\
Should you encounter further problems it is possible, that you need to install the  python packages rrdtool and pymongo. You can install these by running:\\
\begin{lstlisting}
sudo apt-get install python-rrdtool python-pymongo
\end{lstlisting}

	\subsubsection{Ros-kinetic-rosjava-buildtools}
	While running the rosdep install command, it is possible that the ros-kinetic-rosjava-buildtools package cannot be installed and an error occurs. In that case running the command again should fix the problem.


	\subsection{NLP}
	\subsubsection{Sling}
	While installing sling, it is possible that an error occurs, that says the file requirements.txt is missing. In that case you can try downloading sling \href{https://pypi.org/project/sling/#files}{here}. After downloading it extract the files, open sling-0.1.0/sling.egg-info and copy the requires.txt file into sling-0.1.0. Rename it to requirements.txt and open the folder, where you extracted the files. Then run \\
	\begin{lstlisting}
sudo -H pip3 install sling-0.1.0/
sudo -H pip3 install simplenlg tinyrpc==0.9.4 zmq
\end{lstlisting}

	\subsection{Perception}
	\subsubsection{Caffe}
	
	If you encounter the following problem while using your Caffe installation:

\begin{lstlisting}
fatal error: caffe/proto/caffe.pb.h: No such file or directory
\end{lstlisting}

you can fix it by executing the following commands in your caffe directory:


\begin{lstlisting}
protoc src/caffe/proto/caffe.proto --cpp_out=.
mkdir include/caffe/proto
mv src/caffe/proto/caffe.pb.h include/caffe/proto
\end{lstlisting}
	
	\subsection{Manipulation}
	\subsubsection{Python dependencies}
	If you encounter problems, while installing the python packages you can try to update pip by running:\\
	\begin{lstlisting}
sudo pip install --upgrade pip
\end{lstlisting}

	\subsubsection{Object\_state\_listener}	
	Should you encounter the following error, while building the Manipulation package object\_state\_listener:
	\begin{lstlisting}
Could not find a package configuration file provided by "knowrob_objects"
\end{lstlisting}
you can fix it by executing the following steps.\\
\\	

If you dont have the Knowledge part installed and dont want to install it, clone KnowRob outside of the workspace using the following command:\\
\begin{lstlisting}
git clone %*\url{https://github.com/JeremiasThun/knowrob.git}*)
\end{lstlisting}
and copy the knowrob\_objects folder into the src folder of your Manipulation workspace.\\
Otherwise you should install Knowledge first and open the workspace, where you installed it.\\
Now you can copy the knowrob\_objects folder from the directory \\
src/knowledge\_deps/knowrob into the src folder of your Manipulation workspace.\\
\\
 After that is done, open the knowrob\_objects folder in your Manipulation workspace and delete everything but the msg folder and the files CMakeLists.txt, package.xml and README.md.\\
Open the file CMakeLists.txt with a text editor and delete everything except this:\\
\begin{lstlisting}
cmake_minimum_required(VERSION 2.8.3)
project(knowrob_objects)

find_package(catkin REQUIRED COMPONENTS geometry_msgs message_generation roscpp roslib)


add_message_files(FILES ObjectState.msg ObjectStateArray.msg)
generate_messages(DEPENDENCIES geometry_msgs)

catkin_package(CATKIN_DEPENDS roscpp roslib)
\end{lstlisting}
In the next step open package.xml with your text editor and delete everything but this:\\
\begin{lstlisting}
<package>
  <name>knowrob_objects</name>
  <version>1.0.0</version>
  <description>Reasoning about perceived objects.</description>
  <maintainer email="danielb@cs.uni-bremen.de">Daniel Be%*ß*)ler</maintainer>

  <license>BSD</license>

  <url type="website">http://ros.org/wiki/knowrob_objects</url>
  <url type="bugtracker">https://github.com/knowrob/knowrob/issues</url>

  <author>Moritz Tenorth</author>

  <buildtool_depend>catkin</buildtool_depend>

  <build_depend>message_generation</build_depend>
  <build_depend>roslib</build_depend>
  <build_depend>roscpp</build_depend>
  <build_depend>geometry_msgs</build_depend>
  
  <run_depend>message_runtime</run_depend>
  <run_depend>roslib</run_depend>
  <run_depend>roscpp</run_depend>

</package>
\end{lstlisting}
This should fix the problem, so you can try to run
\begin{lstlisting}
catkin build
\end{lstlisting}
in your workspace again.
\end{document}
