\documentclass[main.tex]{subfiles}

\begin{document}

	\begingroup

	\renewcommand{\cleardoublepage}{}

	\renewcommand{\clearpage}{}

	\chapter{Ideas and recommendations for the future}
		
		\section{In general}
				\chapterauthor{}
		Here comes a compact conclusion about what decisions were or were not good and should or should not be done the same way in the future.
		
		\section{Specific features}
		Here comes a list and an explaination of different features that would be nice to implement even though we didn't have the time to implement them ourselves in the end.
	  	
	  	\subsection{Knowledge: Optimize next\_object/1}
	  	\chapterauthor{Jeremias Thun}
	  	The decision which object should be the next one to grasp is still very rudimentary. The responsible predicate \texttt{next\_object/1} in \texttt{pickup.pl} just looks for the nearest object right now (as described in section \ref{sec:kn_pickup}).\\
	  	While that is a good start, it would be nice to mind some other criteria.
	  	
	  	Based on the new object placing strategy it would be pretty nice to weigh in which object fits the best to it's reference object. Especially when you expect to get more information during your run it can significantly improve the resulting order of the objects if the ones that will more likeley be put next to their reference object would be taken earlier.
	  	
		On the other hand if the time is limited and your most important goal is to get at least some of the recognized objects in their final place (this is pretty much the RoboCup scenario), it can be a good idea to take those objects first, that perception was most confident about. This way, if you fail to place all the objects before the time runs out, you will have at least placed the ones you did with a higher confidence. This is a scenario, in which the idea to take the most similar objects first can be a good idea, too: That way, the few actually placed objects would be in an intuitively good order.
		
		If there are significant differences, you could also make some tests and see which objects are easier to grasp so you can take them first. But in our test series there were no significant differences in that matter. 
		
		With all these criteria the real challenge is to make them work with each other without having to use a machine learning algorithm. A nice approach could be for example, to first look for the distance to the robot, define a threshold within wich the physical distance doesn't make a real difference, look for other objects with the same distance to the robot plus this threshold and sort them by one or two of the other criteria. 
		

	\endgroup

\end{document}
