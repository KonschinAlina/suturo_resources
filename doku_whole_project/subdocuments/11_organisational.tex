\documentclass[main.tex]{subfiles}

\begin{document}

	\begingroup

	\renewcommand{\cleardoublepage}{}

	\renewcommand{\clearpage}{}

	\chapter{Organizational difficulties and how we managed them}

		\chapterauthor{Fabian Rosenstock}
		
		\section{Team}
		We tried to organize our work using the Scrum development model in combination with Kanban as explained under \ref{sec:modelofdevelopment}. But in the end, we did not stick to strict patterns and rather used these tools as we saw fit for the given situation. This was partly caused by lacking experience, as we could not properly estimate how long it would take to create or test a feature and did not have a lot of experience working on a project like this.
		While integrating we sometimes encountered problems with components of other groups. Because of this, we made an agreement, that at least one person in each group has to be responsible for the communication with the other groups. Since that person has a better overview of the work of the other groups and has less time for his work it would probably be beneficial to work out a proper system next time.
		
		\section{Milestones}
		
		\subsection{Milestone 1}
		During the first milestone, we tried to establish structures for our work together and communication. As described in the previous section, we didn't follow strict procedures and only really created or enforced strict structures, when something didn't work out. Partly because of this and partly because we had to familiarize ourselves with the system, we faced some communication problems and were generally a bit unorganized. To solve this problem we decided to assign leaders for each expert group.
		
		\subsection{Milestone 2}
		The communication worked better during the second milestone and we reached almost all of our goals. Even though the teamwork did work better, we still missed a few of the deadlines and the distribution of tasks was uneven from time to time. Furthermore, we were asked to do a spontaneous demonstration during this milestone, for which we could not provide a fully functional plan. Because of this, we decided, that every group needed to have a branch with a working version of their code, that was tested with every other part.
		
		\subsection{Milestone 3}
		Clear distribution of tasks leads to easier communication and a more even workload during this milestone.
		A huge organizational challenge was the Coronavirus outbreak. How we handled that is described in the next section. 
		
		\section{Corona virus}
	  	While we were working on milestone 3, we had to make major changes due to the Coronavirus outbreak. The implementation of these changes did cost us some time. During this time it was hard for some to work at all since they worked closely with the robot, which was not possible.
	  	Especially for the groups, that worked close to the robot, the changes lead to a huge loss of motivation.
	  	Despite that, we managed to properly organize our work and got a lot done.
	  	Since we now worked on one project computer we even got a better overview of the work of the other groups and it was easier to get help, after encountering a problem, since everyone worked from home.

	\endgroup

\end{document}


