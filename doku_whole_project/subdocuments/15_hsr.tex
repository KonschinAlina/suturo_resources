\documentclass[main.tex]{subfiles}
\begin{document}
	\chapter{Handling the HSR}
	\chapterauthor{Torge Olliges}
	\label{workin_hsr}
	
	Because the HSR is a machine which could potentially harm people or itself and damage its environment it is important to follow a few rules and to know how to handle the HSR properly. In this chapter it will be explained how to start the HSR in the HSR Lab, how to handle the HSR properly, how to ensure safty of the HSR and its environment and the experiences made during the course of this project.

	\section{Starting the HSR}
	To start the HSR it has to be charged or plugged into the power supply.
	\subsection{Step by step:}
	\begin{itemize}
		\item Turn on the main switch of the HSR
		\item Press the power button of the HSR for approximately 3 seconds (or until the ventilators of the HSR can be heard)
		\item Pull up the emergency switch (if you are using the remote pull up that switch aswell)
		\item Wait for the HSR started speech output (if there is no speech output push the emergency switch down and pull it up again after waiting a brief moment)
		\item Wait until the HSR logo comes up on the head display
		\item The HSR can be used now
	\end{itemize}
	\section{Stopping the HSR}
	\begin{itemize}
		\item Push down the emergency switch on the HSR or the remote
		\item Wait for the Killing all the nodes speech output
		\item If the HSR was plugged into the power supply toggle the switch of the power supply to be turned of before shutting down the HSR
		\item Press down the power button of the HSR for approximately 3 seconds or until the ventilators of the HSR can no longer be heard
		\item Toggle the power switch of the HSR
	\end{itemize}
	\section{Handling the HSR}
	\subsection{While in movement}
	While the HSR is moving it should be avoided to step into the way of the HSR. This could not just stop the execution of the current actions it could also potentially harm the HSR or the one stepping into its way.
	It should be avoided to let the HSR move with its arm extended or its core in the uppermost position as this can lead to damaging the HSR if it tips over or collides with its arm. So it would be best to move the HSR only in its neutral position although this can be disregarded if the task requires differently.
	It is also important to know that the HSR can only avoid objects which it sees. As the robots laser scanner does not scan $360^\circ$, only a range in front of itself it can generally not see obstacles while driving backwards.
	If the HSR is in movement it should be observed with care if it might collide with obstacles and if it is unsure if the HSR will collide or not it is better to prevent the movement as it could inflict damage to the robot or the objects it collides with.
	Any obstacles which are to small for the robot to detect or which can end up under the robot (e.g. power cable of the HSR if plugged in) have to be moved aside as this can damage the movement mechanics of the HSR.
	\subsection{While standing still}
	The crevices and joint regions of the robot should not be touched as it can pinch the one handling the robot incorrectly(e.g. neck region, arm joints et cetera).
	The robot should not hold an object for a very long time (e.g. over 30 minutes) as this puts a lot of stress on the gripper joint and can lead to damage by overheating.
	\section{Other}
	As the HSR has sensors for magnetic strips in its base this can lead to problems if the robot is traversing wire channels.
\end{document}
