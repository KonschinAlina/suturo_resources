\documentclass[main.tex]{subfiles}
\begin{document}
	
	\chapter{Navigation}
		\chapterauthor{Marc Stelter}
		\section{Mapping}
		
		The mapping process for the HSR is already well covered by it's own documentation
		\ \\
		In order to make the process easier to tweak a launchfile has been created for hector\_slamm. It simply starts a hector\_mapping node with some predefined parameters. At the moment these are the default parameters suggested by the original documentation.
		
		\begin{lstlisting}
<launch>
  <node pkg="hector_mapping" type="hector_mapping"
   name="hector_mapping" output="screen">
    <param name="map_size" value="2048"/>
    <param name="map_resolution" value="0.05"/>
    <param name="pub_map_odom_transform" value="true"/>
    <param name="scan_topic" value="/hsrb/base_scan"/>
    <param name="use_tf_scan_transformation" value="true"/>
    <param name="map_update_angle_thresh" value="2.0"/>
    <param name="map_update_distance_thresh" value="0.10"/>
    <param name="scan_subscriber_queue_size" value="1"/>
    <param name="update_factor_free" value="0.39"/>
    <param name="update_factor_occupied" value="0.85"/>
    <param name="base_frame" value="base_link"/>
  </node>
</launch>
		\end{lstlisting}
		
		\section{Localization}
		The original localization of the robot has to be done manually at the moment.
		\ \\
		the snap\_map\_icp package has not yet been included in the process and is left for future tests and validation.
		
		\section{Movement}
		The included move\_base action client is used to move the robot to a given goal. The goal is defined with a move\_base\_ simple goal action. The move\_base package then finds a path based on local and global path finding based on corresponding costmaps. 
	
\end{document}