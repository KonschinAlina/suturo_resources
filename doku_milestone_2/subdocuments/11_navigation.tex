\documentclass[main.tex]{subfiles}
\begin{document}
	
	\chapter{Navigation}
	\chapterauthor{Marc Stelter}
		\section{General}
		During this Milestone the main task of Navigation has been to find Objects on the floor.
		
		\section{Finding objects}
		The final solution during this milestone for the task has been to create a new python node, which interprets the laser scan and Map too publish an PoseArray of points that might be an object of interest. The points are published as PoseArray to enable easy visualization of the results in rviz while containing a minimal overhead in the form of an orientation and no data loss in the form of position.
		
		\subsection{Algorithm}
		The algorithm to find these obstacles works in three steps. First of all the rays of the laser scan get bundled as clusters. This is archived by calculating the point of a ray on the map and checking if the point is within a certain distance to the previous ray if this is true the ray gets added to that cluster. Otherwise a new cluster is started.
		
		In the second step all clusters, which do not consist of at least a minimum number or rays, are removed from further processing.
		
		During the last step for each point of a cluster is compared with the map. Another tolerance parameter is used here. If a certain percentage of points is matched with an obstacle (defined as an occupancy threshold) on the map the cluster gets removed.
		
		For the remaining clusters the center of each one is calculated and returned as pose.
		
		\subsection{Parameters}
		The values for the parameters have been derived through testing and are as follow:
		\begin{itemize}
			\item occ\_threshold: 40 
			\item min\_scans\_cluster: 10
			\item min\_percentage\_covered: 0.7
			\item rad\_neighbour: 0.1 (m)
			\item error\_map: 0.1 (m)
			\item use\_every\_n\_scan: 15
		\end{itemize}
		
		\subsection{Problems} 
		While the above algorithm is an easy and stable solution for the initial task a problem originating in the hardware of the hsr remains. The laser scanner is positioned at an height of 15 cm, which makes the detection of all objects below this point impossible.
	
	
\end{document}