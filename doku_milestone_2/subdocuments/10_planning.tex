\documentclass[main.tex]{subfiles}
\begin{document}
	
	\chapter{Planning}
		\chapterauthor{Torge Olliges, Tom-Eric Lehmkuhl}
		
                The planning team is consisting of Torge Olliges (Groupleader), Philipp Klein (Clean up), Tom-Eric Lehmkuhl (Grocery storing) and Jan Schimpf (Clean up). The group is responsible for the integration and connection of the results of the other groups. To achieve this task the planning group orientated itself along the lines of the legacy code of our predecessors and came up with the general architecture of the project. The planning group was also mainly in charge of testing on the HSR. At the beginning of this milestone planning created diagrams which model how the robot has to behave, depending on the task (Clean up/Grocery storing). During the time that the group worked on the second milestone, planning communicated the progress and current tasks in the group so that everybody knew what the other group members were doing and how they were progressing. In the following sections (one for each package or abstraction layer) the concrete tasks and results of our work will explained and presented.
		
                \section{Low Level Interfaces (LLIF)}
                \subsection{Manipulation}
                \subsubsection{New}
                \subsubsection{Modified}
The /textbf {Grasp-action-client} and textbf {Place-action-client} were updated with an additional argument  grasp pose which specifics from which direction the object is grasped.
The /textbf {percieve-action-client} renamed to /textbf {take-pose-action-client} to prevent further confusion between it and the perception client.
The /textbf {take-pose-action-client} was reworked and now either takes a Int for a set pose or it gets the value 0 to and then ten more Int's to move the HSR into a pose.
                \subsubsection{Unchanged}
                \subsection{Navigation}
                \subsubsection{New}
                \subsubsection{Modified}
                \subsubsection{Unchanged}
                \subsection{Knowledge}
                \subsubsection{New}
                \begin{itemize}
 				          \item knowlegde-inserter is an interface to knowledge, which can insert the objects recognized by perception into the knowledge base
				        \end{itemize}
				          \textbf{knowledge-client} \\
				          The knowledge-client is an interface to knowledge, which is responsible for the request to the knowledgebase. For every request exists a function, which calls the prolog-query and returns the result. Implemented requests are:
				          \begin{itemize}
				             \item \textit{prolog-table-objects} returns all objects on the table
				             \item \textit{prolog-object-goal} returns the goal shelf for an object
				             \item \textit{prolog-object-goal-pose} returns the goal pose for an object
				             \item \textit{prolog-all-objects-in-shelf} returns all objects in the shelf
				             \item \textit{prolog-next-object} returns the next object to grasp choosen by knowledge
				             \item \textit{prolog-object-dimensions} returns the dimension for an object (depth, width, height)
				             \item \textit{prolog-object-pose} returns the pose of an object as list
				             \item \textit{prolog-table-pose} returns the pose of the table as list (x, y, z)
				             \item \textit{prolog-shelf-pose} returns the pose of the shelf as list (x, y, z)
				             \item \textit{prolog-object-in-gripper} returns the dimension of the object in gripper as list (depth, width, height)
				           \end{itemize} 
				            The functions \textit{knowledge-set-table-source} and \textit{knowledge-set-ground-source} are for setting the sources for selecting the next object. The task "Storing Groceries" needs the table as source and "Clean-up" needs also the ground as source. 
				        
                \subsubsection{Modified}
                \subsubsection{Unchanged}
                \subsection{Perception}
                \subsubsection{New}
                \subsubsection{Modified}
                \subsubsection{Unchanged}
                \subsection{NLP}
                \subsubsection{New}
                The \textbf{nlp-subscriber} provides functions for subscribing the topics from nlp. At the topic \textit{suturo\_speech\_recognition\/hard\_commands} the commands "start", "stop" and "continue" are published, if they are recognised by nlp. The subscriber for the hard-commands \textit{static-command-listener} calls the function \textit{set-state-fluent} with the message as argument and a fluent-variable is set according to the command. The message for dynmaic-commands is not defined yet.
                \subsubsection{Modified}
                \subsubsection{Unchanged}

                
                \section{Common Functions (COMF)}
                The common functions package consists of functions which are common to both tasks. We created a functions file corresponding to the groups which they are interacting with and providing functionality to.
                \subsection{Designators}
                For this milestone we added designators to use in our high level plans.
                \subsubsection{New} 
                The designators \textit {grasping} and \textit {placing} are wrappers for \textit {call-grasp-action} and \textit {call-place-action} and both get 12 arguments the x,y,z coordiants of the object, the x,y,z,w quaternion values, the x,y,z dimensions of the object, the object-id and the grasp-pose. 

                \subsubsection{Modified}
                \subsubsection{Unchanged}
                
                \subsection{Manipulation}
                \subsubsection{New}
                Manipulation Functions:
                grasp-object
                place-object
                place-object-list
                create-place-list
                
                
                
                                  \begin{itemize}  
                  \item \textit{grasp-object} in \textbf{manipulation-functions} gets a object-id and the grasp-pose as arguments, uses the object-id querie knowledge for the pose of the object with \textit{prolog-object-pose}, the dimensions of the object with \textit{prolog-object-dimensions} and then uses the returned data to call the grasping designator.
                  \item \textit{place-object} in \textbf{manipulation-functions} gets the object-id of the object in the gripper and which grasp-pose is used to grasp it as arguments, then knowledge is queried to get the goal pose of the object ith \textit{prolog-object-goal-pose}, the dimensions of the object with \textit{prolog-object-dimensions} of the object and then uses the returned data to call the placing designator.
                  \item \textit{place-object-list} in \textbf{manipulation-functions}
                  gets a list with arguments to call the placing designator 
                  \item \textit{create-place-list} in \textbf{manipulation-functions} 
                  gets the object-id of the object in the gripper and which grasp-pose is used to grasp it as arguments, then knowledge is queried to get the goal pose of the object ith \textit{prolog-object-goal-pose}, the dimensions of the object with \textit{prolog-object-dimensions} of the object, this is then used to create a list with five elements that can be used to call \textit{place-object-list}. Four of these elements are slightly offset on either the x-coordinate or the y-coordinate by 5cm.  
                  \item \textit{place-hsr} in \textbf{high-level-plans}  gets the object-id of the object in the gripper and which grasp-pose is used to grasp it as arguments, makes use of \textit{create-place-list}, calls \textit{place-object-list} with the first element, if it fails then the first element is removed from the list and retried and loop until either we had a success or the list is empty
                  \item \textit{grasp-hsr} in \textbf{high-level-plans} gets a object-id and the grasp-pose as arguments, it calls \textit{place-object} and if that fails it will get into position to percieve the object, call perception and then insert the new data into knowledge and then retry the call \textit{place-object} for the object.
                  \item \textit{move-grasp} in \textbf{high-level-plans} gets a object-id and the grasp-pose as arguments, it moves the hsr into a position from where it can grasp and then calls \textit{grasp-hsr }.
                  
                  \end{itemize}
                \subsubsection{Modified}
                \subsubsection{Unchanged}
                
                \subsection{Navigation}
                \subsubsection{New}
                  \begin{itemize}
                    \item \textit{move-to-table} in \textbf{high-level-plans} queries the position of the table from knowledge, and adds a buffer, to create a position before the table and navigates the robot to it. If the argument is NIL, the robot will position himself frontally to the table. If the argument is set to T, the robot will be turned for percieving the table.
                    \item \textit{move-to-shelf} in \textbf{high-level-plans} queries the position of the shelf from knowledge, and adds a buffer, to create a position before the shelf and navigates the robot to it. If the argument is NIL, the robot will position himself frontally to the shelf. If the argument is set to T, the robot will be turned for perceiving the shelf.
                  \end{itemize}
                \subsubsection{Modified}
                \subsubsection{Unchanged}
                
                \subsection{Knowledge}
                \subsubsection{New}
                \subsubsection{Modified}
                \subsubsection{Unchanged}Knowledge Functions:
                
                \subsection{Perception}
                \subsubsection{New}
                \subsubsection{Modified}
                \subsubsection{Unchanged}Perception Functions:
                
                \subsection{NLP}
                \subsubsection{New}
                \subsubsection{Modified}
                \subsubsection{Unchanged}NLP functions:
                
                \section{Grocery Storing (GROCERY)}
                This package depends on common functions and low level interfacing and contains the execute file for the grocery task. Furthermore it contains a init file for the task. This connects our action clients out of the LLIF package to the action servers of the other groups and starts the master node for planning (planning\_node). The grocery\_bw file provides functions for the interaction with the bullet world.   
                \section{Cleanup (CLEAN)}
                Equal to GROCERY just different inits and execute files.
                \section{Bullet World}
                The bulletworld is a new feature which we implemented for this milestone. It is a representation of the real world which lets us check for visibilty or reachability without having to use the real robot. A function to spawn primitive objects in the bullet world has been added, which was detected by the perception part to have a representation of the current world state. It was decided not to test grasping with the bullet world because giskard can't be used in the bullet world, this would minder planning's capabilities. 
                
\end{document}
