\documentclass[main.tex]{subfiles}

\begin{document}

	

	\chapter{NLP/NLG}
	\chapterauthor{Merete Bommarius, Paul Schnipper}
  
  \section{Natural Language Generation}
  
  
  \section{Context-Free Grammar}
  NLP works with something called context-free grammar (CFG). It is a certain type of formal grammar, meaning, a set of  production ruled that describe all possible strings in a given language. Productions rules are simple replacements. For example there is a sentence like:

\begin{equation}
Bring the [item] to the [surface].
\end{equation}

Both [item] and [surface] are replacable and would count as a production rule. Now, this example is already more specific. A more vague example would look like this:

\begin{equation}
A → α
A → β
\end{equation}

A can be replaced by about anything given any value. To translate this into a more Suturo related context, the grammar would translate to this:

\begin{equation}
A → [item]
A → [surface]
\end{equation}

Both item and surface can then be further defined. In CFG all rules are one-to-one, one-to-many, and one-to-none. These rules are applied regardless of context, hends the name context-free grammar. A nother less official rule is, that the production rule always is a nonterminal symbol, but for understanding purposes, this inofficial rule was ignored here.

\end{document}
