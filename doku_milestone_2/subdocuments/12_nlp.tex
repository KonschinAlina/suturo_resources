\documentclass[main.tex]{subfiles}

\begin{document}

	

	\chapter{NLP/NLG}
	\chapterauthor{Merete Bommarius, Paul Schnipper}
  
  \section{Natural Language Generation}
  Natural-language generation (NLG) is a software process that transforms structured data into natural language. It can be used to produce automated reports for companies, as well as custom content for a web or mobile application.  NLG is also used in well-known applications like Google or Amazon‘s Alexa
Automated NLG can be compared to the process humans use when they turn ideas into writing or speech. This is referred to as language production.
To make the grammar more correct more easily, the current program imports a software called simplenlg. It helps with sentence structures and for example tenses and to distinguish between singular and plural. 
The output the robot can give if it was given a task is split into six different sections. These sections are called; reject, task, waiting, failed, finished, and complications. Reject describes the decision the robot makes to deny the task, while task described the decision to perform it. Waiting and failed are similar. While waiting describes the robot waiting to understand the given task, failed describes that the robot has failed to understand. Finished means that the robot has finished performing the given task and complications describe, as the name says, complications that can occur during the performance of the task. Outcomes for the different sections can for example be:

\begin{equation}
Reject = “Get someone else.“ or “Me is too superior for task.“
Task = “Yes, Human.“ or “Me got Human covered.“ 
Waiting = “Calibrating.“ or “Please wait.“
Failed = “Please speak loud and clear.“ or “Please repeat.“
Finished = “Mischief managed“ or “Task completed“
Complications = “Humans are too tall, Me cannot see.“ or “Object is too small.“
\end{equation}

The reject section is solely for laughs and giggles. The robot will still perform the task.
  
  \section{Context-Free Grammar}
  NLP works with something called context-free grammar (CFG). It is a certain type of formal grammar, meaning, a set of  production ruled that describe all possible strings in a given language. Productions rules are simple replacements. For example there is a sentence like:

\begin{equation}
Bring the [item] to the [surface].
\end{equation}

Both [item] and [surface] are replacable and would count as a production rule. Now, this example is already more specific. A more vague example would look like this:

\begin{equation}
A \rightarrow \alpha
\end{equation}
\begin{equation}
A \rightarrow \beta
\end{equation}

A can be replaced by about anything given any value. To translate this into a more Suturo related context, the grammar would translate to this:

\begin{equation}
A \rightarrow [item] 
\end{equation}
\begin{equation}
A \rightarrow [surface]
\end{equation}

Both item and surface can then be further defined. In CFG all rules are one-to-one, one-to-many, and one-to-none. These rules are applied regardless of context, hends the name context-free grammar. A nother less official rule is, that the production rule always is a nonterminal symbol, but for understanding purposes, this inofficial rule was ignored here.

\end{document}
