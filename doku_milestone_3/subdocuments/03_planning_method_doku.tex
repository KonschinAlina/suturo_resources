\documentclass[main.tex]{subfiles}

\begin{document}

	\begingroup

	\renewcommand{\cleardoublepage}{}

	\renewcommand{\clearpage}{}

	\chapter{Planning Function Documentation}

		\chapterauthor{Planning Group}
		
		\section{GROCERY}
		
		\subsection{execute-grocery.lisp}
		This file contains the main function for this tasks and some subroutines.
		\begin{itemize}
			\item \textbf{execute-grocery} \\
			This is the main function for the grocery storing task, when this function is called it starts all subroutines to execute the planned behavior for this task, including failure handling
			\item \textbf{perceive-shelf} \\
			This function is a subroutine for perceiving shelfs depending on the given region value
			\item \textbf{perceive-table} \\
			This function is a subroutine for perceiving the table depending on the table position given by the knowledge base
			\item \textbf{grasp-handling} \\
			This function is a subroutine for grasping objects with primitive failure handling, which object will be grasped is determined by the knowledge base
			\item \textbf{grasp-with-failure-handling} \\
			This function is a subroutine for grasping objects with more sophisticated failure handling, which object will be grasped is determined by the knowledge base
			\item place-handling
			This function is a subroutine for placing objects
		\end{itemize}
		
		\subsection{grocery-bw.lisp}
		This file contains functions used in combination of the bulletworld simulation.
		\begin{itemize}
			\item \textbf{spawn-btr-objects}
			This function spawns given objects as primitiv objects in the bulletworld simulation at the corresponding position of the by robosherlock perceived real life objects 
		\end{itemize}
	  	
	  	\section{CLEANUP}
	  	\subsection{execute-cleanup.lisp}
	  	\begin{itemize}
			\item \textbf{execute-cleanup} \\
			This is the main function for the clean up task, when this function is called it starts all the subroutines to execute the planned behavior for this task
			\item \textbf{shelf-scan} \\
			perceives the entire shelf using perceive-shelf
			\item \textbf{perceive-shelf} \\
			same as in grocery perceives the shelf depending on the given regions value
			\item \textbf{perceive-table} \\
			same as in grocery perceives the robocub table
			\item \textbf{transport} \\
            transports all objects found on the robocub table and then places them into the shelf
			\item \textbf{point-of-interest-search} \\
			moves to a point of interest and perceives it 
			\item \textbf{point-of-interest-transport} \\
			grasp objects found with point of interest search and places them into the shelf
			\item \textbf{grasp-with-failure-handling-floor} \\
			grasping with failure handling it falls to grasp an object from the floor
			\item \textbf{grasp-handling}\\
			grasping with primitive failture handling for grasping from the robocub table
		\end{itemize}

	    \subsection{cleanup-bw.lisp}
		\begin{itemize}
			\item \textbf{spawn-btr-objects}
			This function spawns given objects as primitiv objects in the bulletworld simulation at the corresponding position of the by robosherlock perceived real life objects  
		\end{itemize}
	  	
	  	\section{COMF}
	    \subsection{high-level-plans.lisp}
	    This file contains all high level plans.(high level plans ar below execute level)
	    \begin{itemize}
	    \item \textbf{try-movement} \\
	    
		\item \textbf{try-movement-stampedList} \\
		tries a given list of
	    \item \textbf{move-hsr} \\
	    This function takes a stamped pose as input and moves the robot to the given position if the position can be reached
	    \item \textbf{grasp-hsr} \\
	    FEHLT
	    \item \textbf{place-hsr} \\
	    FEHLT
	    \item \textbf{move-to-poi} \\
	    This function moves the robot to the next point of interest that can be approached. 
	    \item \textbf{move-to-poi-and-scan} \\
	    This function moves the robot to the next point of interest that can be approached, scans all objects on the ground and then turns back towards the object.
	    \item \textbf{move-to-table} \\
	    This function moves the robot to the table position which is determined by the knowledge base, it takes a boolean value which defines if the robot should be in perceiving or placing/grasping base pose at the goal position (nil = grasping/placing, T = perceiving) 
	    \item \textbf{move-to-shelf} \\
	    This function moves the robot to the shelf position which is determined by the knowledge base, it takes a boolean value which defines if the robot should be in perceiving or placing/grasping base pose at the goal position (nil = grasping/placing, T = perceiving) 
		\end{itemize}
	    \subsection{knowledge-functions.lisp}
	    The knowledge function file contains all functions common to both tasks which are used in the connection with the knowledge part
	    \begin{itemize}
	    	\item \textbf{Fehlt} \\
	    \end{itemize}
	    \subsection{manipulation-functions.lisp}
	    The manipulation function file contains all functions common to both tasks which are used in the connection with the manipulation part
	    \begin{itemize}
	    \item \textbf{place-object} \\
	    gets the object id and the grasp position to create a place designator  
		\item \textbf{place-object-list} \\
		gets a list and uses that to create a place designator 
	    \item \textbf{grasp-object} \\
	    gets the object id and what grasp position should be used to create a grasp desingator
	    \item \textbf{create-place-list} \\
	    gets the object id and grasp position to create a list with positions to place the object
		\end{itemize}
	    \subsection{navigation-functions.lisp}
	    The navigation function file contains all functions common to both tasks which are used in the connection with the navigation part
	    \begin{itemize}
	    	\item \textbf{points-around-point} \\
	    	This function returns a given number of positions with a given 	distance to a point. The positions are either rotated 90 degrees to the point, or point directly to the object.
	    	\item \textbf{pose-with-distance-to-points} \\
	    	This function moves the robot with a given distance to a given point. The positions are either rotated 90 degrees to the point, or point directly to the object. It will only move to positions that are not in the obstacle map and are accessible in the simulator.
	    	\item \textbf{pointInPolygon} \\
This function checks if a point is inside a polygon.
	    \end{itemize}
	    \subsection{nlp-functions.lisp}
	    \begin{itemize}
	    	\item \textbf{Fehlt} \\
	    	Ist vielleicht sogar leer... dann file löschen?
	    \end{itemize}
	    \subsection{perception-functions.lisp}
	    The perception function file contains all functions common to both tasks which are used in the connection with the perception part
	    \begin{itemize}
	    	\item \textbf{get-confident-objects} \\
	    	This function filters the objects detected by robosherlock by their confidence (class, shape, color), the default threshold value is 0.5
	    \end{itemize}
	    \subsection{safety-check.lisp}
	    \begin{itemize}
	    	\item \textbf{Fehlt} \\
	    \end{itemize}
	    \subsection{process-modules.lisp}
	    \begin{itemize}
	    	\item \textbf{Fehlt} \\
	    \end{itemize}
	    \subsection{select-process-modules.lisp}
	    \begin{itemize}
	    	\item \textbf{Fehlt} \\
	    \end{itemize}
	    \subsection{designators.lisp}
	    \begin{itemize}
	    	\item \textbf{Fehlt} \\
	    \end{itemize}

	  	\section{LLIF}
		Hier fehlen evtl noch files:
		\subsection{grasp-action-client.lisp}
		\subsection{place-action-client.lisp}
		\subsection{take-pose-action-server.lisp}
		\subsection{move-gripper-client.lisp}
		\subsection{knowledge-client.lisp}
		\subsection{knowledge-insertion-client.lisp}
		\subsection{navigation-action.lisp}
		\subsection{nlp-subscriber.lisp}
		\subsection{text-to-speech.lisp}
		\subsection{obstacle-map-subscriber.lisp}
	    \begin{itemize}
	    	\item \textbf{obstacle-map-listener} \\
	    	This function subscribes the obstacle map and stores the current map in a variable.
	    	\item \textbf{saveObstacleMap} \\
	    	Makes a copy of the given message and stores it in a variable.
	    	\item \textbf{getMapPoint} \\
	    	Returns the occupancy from a given position in the map.
	    \end{itemize}
		\subsection{poi-client.lisp}
 		\begin{itemize}
	    	\item \textbf{closestPoi} \\
	    	This function returns the next point of interest in relation to the passed point.
	    	\item \textbf{sortedPoiByDistance} \\
	    	This function returns all points of interest sorted in relation to the given point.
	    	\item \textbf{point-listener} \\
	    	This function subscribes the point of interessts and stores them in a variable.
	    \end{itemize}
		\subsection{robosherlock-client-object.lisp}
		This file is responsible for the low level communication with the robosherlock pipeline which is responsible for objects.
		\begin{itemize}
			\item \textbf{init-robosherlock-object-action-client} \\
			This function initiates the connection to the robosherlock object action server
			\item \textbf{get-robosherlock-client} \\
			This function returns the robosherlock action client
			\item \textbf{make-action-goal} \\
			This function makes and returns a action client goal
			\item \textbf{call-robosherlock-object-pipeline} \\
			This function takes a array of regions and a boolean value for visualization, this function calls the robosherlock object pipeline with the input values and returns the result
		\end{itemize}
		\subsection{robosherlock-client-plane.lisp}
		This file is responsible for the low level communication with the robosherlock pipeline which is responsible for planes e.g. doors.
		\begin{itemize}
			\item \textbf{init-robosherlock-plane-action-client} \\
			This function initiates the connection to the robosherlock plane action server
			\item \textbf{get-robosherlock-client} \\
			This function returns the robosherlock action client
			\item \textbf{make-action-goal} \\
			This function makes and returns a action client goal
			\item \textbf{call-robosherlock-plane-pipeline} \\
			This function takes a array of regions and a boolean value for visualization, this function calls the robosherlock object pipeline with the input values and returns the result
		\end{itemize}
	\endgroup

\end{document}
