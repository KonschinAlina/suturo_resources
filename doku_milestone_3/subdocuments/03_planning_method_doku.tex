\documentclass[main.tex]{subfiles}

\begin{document}

	\begingroup

	\renewcommand{\cleardoublepage}{}

	\renewcommand{\clearpage}{}

	\chapter{Planning Function Documentation}

		\chapterauthor{}
		
		\section{GROCERY}
		
		\subsection{execute-grocery.lisp}
		
		\begin{itemize}
			\item \textbf{execute-grocery} \\
			main function for the grocery storing task when this function is called it starts all subroutines to execute the planned behavior for this task
			\item \textbf{perceive-shelf} \\
			this function perceives shelfs depending on the given region
			\item perceive-table \\
			perceives the table depending on the table position given by the knowledge base
			\item \textbf{grasp-handling} \\
			grasping with primitive failure handling, which object will be grasped is determined by the knowledge base
			\item \textbf{grasp-with-failure-handling} \\
			grasping with more sophisticated failure handling, which object will be grasped is determined by the knowledge base
			\item place-handling
		\end{itemize}
		
		\subsection{grocery-bw.lisp}
		
		\begin{itemize}
			\item spawn-btr-objects
			this functions spawns given primitiv objects at the position of the by robosherlock perceived objects 
		\end{itemize}
	  	
	  	\section{CLEANUP}
	  	\subsection{execute-cleanup.lisp}
	  	\begin{itemize}
			\item \textbf{execute-cleanup} \\
			main function for the clean up task when this function is called it starts all the subroutines to execute the planned behavior for this task
			\item \textbf{shelf-scan} \\
			perceives the entire shelf using perceive-shelf
			\item \textbf{perceive-shelf} \\
			same as in grocery perceives the shelf depending on the given regions value
			\item \textbf{perceive-table} \\
			same as in grocery perceives the robocub table
			\item \textbf{transport} \\
            transports all objects found on the robocub table and then places them into the shelf
			\item \textbf{point-of-interest-search} \\
			moves to a point of interest and perceives it 
			\item \textbf{point-of-interest-transport} \\
			grasp objects found with point of interest search and places them into the shelf
			\item \textbf{grasp-with-failure-handling-floor} \\
			grasping with failure handling it falls to grasp an object from the floor
			\item \textbf{grasp-handling}\\
			grasping with primitive failture handling for grasping from the robocub table
		\end{itemize}

	    \subsection{cleanup-bw.lisp}
		\begin{itemize}
			\item spawn-btr-objects
			this functions spawns given primitiv objects at the position of the by robosherlock perceived objects 
		\end{itemize}
	  	
	  	\section{COMF}
	    \subsection{high-level-plans.lisp}
	    \begin{itemize}
	    \item \textbf{try-movement} \\
	    
		\item \textbf{try-movement-stampedList} \\
		tries a given list of
	    \item \textbf{move-hsr} \\
	    This function takes a stamped pose as input and moves the robot to the given position if the position can be reached
	    \item \textbf{grasp-hsr} \\
	    FEHLT
	    \item \textbf{place-hsr} \\
	    FEHLT
	    \item \textbf{move-to-poi} \\
	    \item \textbf{move-to-poi-and-scan} \\
	    \item \textbf{move-to-table} \\
	    This function moves the robot to the table position which is determined by the knowledge base, it takes a boolean value which defines if the robot should be in perceiving or placing/grasping base pose at the goal position (nil = grasping/placing, T = perceiving) 
	    \item \textbf{move-to-shelf} \\
	    This function moves the robot to the shelf position which is determined by the knowledge base, it takes a boolean value which defines if the robot should be in perceiving or placing/grasping base pose at the goal position (nil = grasping/placing, T = perceiving) 
		\end{itemize}
	    \subsection{knowledge-functions.lisp}
	    \begin{itemize}
	    	\item \textbf{Fehlt} \\
	    \end{itemize}
	    \subsection{manipulation-functions.lisp}
	    \begin{itemize}
	    \item \textbf{place-object} \\
	    gets the object id and the grasp position to create a place designator  
		\item \textbf{place-object-list} \\
		gets a list and uses that to create a place designator 
	    \item \textbf{grasp-object} \\
	    gets the object id and what grasp position should be used to create a grasp desingator
	    \item \textbf{create-place-list} \\
	    gets the object id and grasp position to create a list with positions to place the object
		\end{itemize}
	    \subsection{navigation-functions.lisp}
	    \begin{itemize}
	    	\item \textbf{Fehlt} \\
	    \end{itemize}
	    \subsection{nlp-functions.lisp}
	    \begin{itemize}
	    	\item \textbf{Fehlt} \\
	    \end{itemize}
	    \subsection{perception-functions.lisp}
	    The perception function file contains all functions common to both tasks which are used in the connection with the perception part
	    \begin{itemize}
	    	\item \textbf{get-confident-objects} \\
	    	This function filters the objects detected by robosherlock by their confidence (class, shape, color), the default threshold value is 0.5
	    \end{itemize}
	    \subsection{safety-check.lisp}
	    \begin{itemize}
	    	\item \textbf{Fehlt} \\
	    \end{itemize}
	    \subsection{process-modules.lisp}
	    \begin{itemize}
	    	\item \textbf{Fehlt} \\
	    \end{itemize}
	    \subsection{select-process-modules.lisp}
	    \begin{itemize}
	    	\item \textbf{Fehlt} \\
	    \end{itemize}
	    \subsection{designators.lisp}
	    \begin{itemize}
	    	\item \textbf{Fehlt} \\
	    \end{itemize}

	  	\section{LLIF}
		Fehlt einfach mal komplett?
	\endgroup

\end{document}
