\documentclass[main.tex]{subfiles}

\begin{document}

	\begingroup

	\renewcommand{\cleardoublepage}{}

	\renewcommand{\clearpage}{}

	\chapter{Planning Function Documentation}

		\chapterauthor{Planning Group}
		
		\section{GROCERY}
		
		\subsection{execute-grocery.lisp}
		This file contains the main function for this tasks and some subroutines.
		\begin{itemize}
			\item \textbf{execute-grocery} \\
			This is the main function for the grocery storing task, when this function is called it starts all subroutines to execute the planned behavior for this task, including failure handling
			\item \textbf{perceive-shelf} \\
			This function is a subroutine for perceiving shelfs depending on the given region value
			\item \textbf{perceive-table} \\
			This function is a subroutine for perceiving the table depending on the table position given by the knowledge base
			\item \textbf{grasp-handling} \\
			This function is a subroutine for grasping objects with primitive failure handling, which object will be grasped is determined by the knowledge base
			\item \textbf{grasp-with-failure-handling} \\
			This function is a subroutine for grasping objects with more sophisticated failure handling, which object will be grasped is determined by the knowledge base
			\item \textbf{place-handling}
			This function is a subroutine for placing objects
		\end{itemize}
		
		\subsection{grocery-bw.lisp}
		This file contains functions used in combination of the bulletworld simulation.
		\begin{itemize}
			\item \textbf{spawn-btr-objects}
			This function spawns given objects as primitiv objects in the bulletworld simulation at the corresponding position of the by robosherlock perceived real life objects 
		\end{itemize}
	  	
	  	\section{CLEANUP}
	  	This file contains the main function for this tasks and some subroutines.
	  	\subsection{execute-cleanup.lisp}
	  	\begin{itemize}
			\item \textbf{execute-cleanup} \\
			This is the main function for the clean up task, when this function is called it starts all the subroutines to execute the planned behavior for this task
			\item \textbf{shelf-scan} \\
			This function perceives the entire shelf using perceive-shelf
			\item \textbf{perceive-shelf} \\
			Same as in grocery perceives the shelf depending on the given regions value
			\item \textbf{perceive-table} \\
			Same as in grocery perceives the robocub table
			\item \textbf{transport} \\
            This function transports all objects found on the robocup table and then places them into the shelf
			\item \textbf{point-of-interest-search} \\
			This function moves to a point of interest and perceives it 
			\item \textbf{point-of-interest-transport} \\
			This function grasps objects found with point of interest search and places them into the shelf
			\item \textbf{grasp-with-failure-handling-floor} \\
			This function is a subroutine for grasping objects from the floor with failure handling, which object will be grasped is determined by the knowledge base
			\item \textbf{grasp-handling}\\
			This function is a subroutine for grasping objects with primitive failure handling, which object will be grasped is determined by the knowledge base
		\end{itemize}

	    \subsection{cleanup-bw.lisp}
        This file contains functions used in combination of the bulletworld simulation.
		\begin{itemize}
			\item \textbf{spawn-btr-objects}
			This function spawns given objects as primitiv objects in the bulletworld simulation at the corresponding position of the by robosherlock perceived real life objects  
		\end{itemize}
	  	
	  	\section{COMF}
	    \subsection{high-level-plans.lisp}
	    This file contains all high level plans.(high level plans ar below execute level)
	    \begin{itemize}
		\item \textbf{try-movement-stampedList} \\
		Tries to move the robot to poses from a given list of stamped poses in the bulletworld if one succeeds it returns the succeeding pose 
	    \item \textbf{move-hsr} \\
	    This function takes a stamped pose as input and moves the robot to the given position if the position can be reached
	    \item \textbf{grasp-hsr} \\
	    This function takes a object-id and a grasp-pose and grasps the object corresponding to the object-id and includes failure handling
	    \item \textbf{place-hsr} \\
	    This function takes a object-id and a grasp-pose. Places the object in the gripper in the goal and includes failure handling.
	    \item \textbf{move-to-poi} \\
	    This function moves the robot to the next point of interest that can be approached. 
	    \item \textbf{move-to-poi-and-scan} \\
	    This function moves the robot to the next point of interest that can be approached, scans all objects on the ground and then turns back towards the object.
	    \item \textbf{move-to-table} \\
	    This function moves the robot to the table position which is determined by the knowledge base, it takes a boolean value which defines if the robot should be in perceiving or placing/grasping base pose at the goal position (nil = grasping/placing, T = perceiving) 
	    \item \textbf{move-to-shelf} \\
	    This function moves the robot to the shelf position which is determined by the knowledge base, it takes a boolean value which defines if the robot should be in perceiving or placing/grasping base pose at the goal position (nil = grasping/placing, T = perceiving) 
		\end{itemize}
	    \subsection{knowledge-functions.lisp}
	     We didn't implement any higher level functions for the knowledge interface because they are not needed.
	    \subsection{manipulation-functions.lisp}
	    The manipulation function file contains all functions common to both tasks which are used in the connection with the manipulation part
	    \begin{itemize}
	    \item \textbf{place-object} \\
	    This function gets the object id and the grasp position and uses that to create a place designator  
		\item \textbf{place-object-list} \\
		This function gets a list and uses that to create a place designator 
	    \item \textbf{grasp-object} \\
	    This function gets the object id and what grasp position should be used to grasp the object and uses that to create a grasp desingator
	    \item \textbf{create-place-list} \\
	    This function gets the object id and grasp position to create a list with positions to place the object
		\end{itemize}
	    \subsection{navigation-functions.lisp}
	    The navigation function file contains all functions common to both tasks which are used in the connection with the navigation part
	    \begin{itemize}
	    	\item \textbf{points-around-point} \\
	    	This function returns a given number of positions with a given 	distance to a point. The positions are either rotated 90 degrees to the point, or point directly to the object.
	    	\item \textbf{pose-with-distance-to-points} \\
	    	This function moves the robot with a given distance to a given point. The positions are either rotated 90 degrees to the point, or point directly to the object. It will only move to positions that are not in the obstacle map and are accessible in the simulator.
	    	\item \textbf{point-in-polygon} \\
This function checks if a point is inside a polygon.
	    \end{itemize}
	    \subsection{nlp-functions.lisp}
	     We didn't implement any higher level functions for the nlp interface because they are not needed.
	    \subsection{perception-functions.lisp}
	    The perception function file contains all functions common to both tasks which are used in the connection with the perception part
	    \begin{itemize}
	    	\item \textbf{get-confident-objects} \\
	    	This function filters the objects detected by robosherlock by their confidence (class, shape, color), the default threshold value is 0.5
	    \end{itemize}
	    \subsection{safety-check.lisp}
	    The plan is for the procedure in which the robot is inspected. 
	    \begin{itemize}
	    	\item \textbf{execute-safety-check} \\
	    	Start the plan. The robot moves through the entrance door as soon as it is open. Then it drives through the room to the exit door. The collision detection is checked. In addition, some voice outputs are given out to test that the robot can be heard clearly.
	    	\item \textbf{check-goal-reached}\\
	    	Überprüft, ob der Roboter am Zielpunkt angekommen ist. Dies wird als Workaround benutzt, da der Roboter noch nicht erkennen kann, ob eine Tür geöffnet wurde.
	    \end{itemize}
	    \subsection{process-modules.lisp}
	     \begin{itemize}
	    	\item \textbf{hsr-navigation process module} \\
		A process module which gets a motion deisgnator as input and calls the low level interface to excecute navigation commands.
		\item \textbf{hsr-text-to-speech process module} \\
		A process module which receives a motion-designator and calls the low level interface to exceute speech actions.
		\item \textbf{hsr-perceive process module} \\
		A process module which receives a motion-designator and calls the low level interface to execute perception actions.
		\item\textbf{hsr-arm process module} \\
		A process module which receives a motion-designator and performs grasping or placing actions by calling the low level interfaces to execute those manipulation commands 
	    \end{itemize}
	    \subsection{select-process-modules.lisp}
	     \begin{itemize}
	    	\item \textbf{Fehlt} \\
	    \end{itemize}
	    \subsection{designators.lisp}
	    Contains motion-designators for of the following type
	    \begin{itemize}
	    	\item \textbf{going} \\
		Used for navigation purposes.
		\item \textbf{moving-torso} \\
		Used for moving the torso.
		\item \textbf{looking} \\
		Used for neck movement and general looking purposes.
		\item \textbf{opening} \\
		
		\item \textbf{closing} \\
	    \end{itemize}

	  	\section{LLIF}
		Hier fehlen evtl noch files:
		\subsection{grasp-action-client.lisp}
		This file is responsible for the low level communication with the
		grasp action server which is responsible for grasping objects.
		\begin{itemize}
			\item \textbf{init-grasp-action-client} \\
			This function initiates the connection to the grasp action server
			\item \textbf{get-grasp-action-client} \\
			This function returns the grasp action client
			\item \textbf{make-grasp-action-goal} \\
			This function makes and returns a action client goal
			\item \textbf{ensure-grasp-goal-reached} \\
            This function makes sure the goal was reached
			\item \textbf{call-grasp-action} \\
			This function takes the x,y,z coordiantes, quaternion values, the x,y,z sizes, the object-id of the object that should be grasped and the graspmode.
		\end{itemize}
		\subsection{place-action-client.lisp}
		This file is responsible for the low level communication with the
		place action server which is responsible for placing objects.
	    \begin{itemize}
			\item \textbf{init-place-action-client} \\
			This function initiates the connection to the place action server
			\item \textbf{get-place-action-client} \\
			This function returns the place action client
			\item \textbf{make-place-action-goal} \\
			This function makes and returns a action client goal
			\item \textbf{ensure-place-goal-reached} \\
            This function makes sure the goal was reached
			\item \textbf{call-place-action} \\
			This function takes the x,y,z coordiantes, quaternion values  and the object-id of the object that is placed and the graspmode.
		\end{itemize}
		\subsection{take-pose-action-server.lisp}
		This file is responsible for the low level communication with the
		take pose action server which is responsible taking different poses.
		\begin{itemize}
			\item \textbf{init-take-pose-action-client} \\
			This function initiates the connection to the take pose action server
			\item \textbf{get-take-pose-action-client} \\
			This function returns the take pose action client
			\item \textbf{make-take-pose-action-goal} \\
			This function makes and returns a action client goal
			\item \textbf{ensure-take-pose-goal-reached} \\
            This function makes sure the goal was reached
			\item \textbf{call-take-pose-action} \\
			This function takes a integeter for a set pose or for a custom pose than needs ten more float values to set the joints.
		\end{itemize}
		\subsection{move-gripper-client.lisp}
		This file is responsible for the low level communication with the
		move gripper action client which is responsible for the moving the gripper.
		\begin{itemize}
			\item \textbf{init-move-gripper-action-client} \\
			This function initiates the connection to the move gripper action server
			\item \textbf{get-move-gripper-action-client} \\
			This function returns the move gripper action client
			\item \textbf{make-move-gripper-action-goal} \\
			This function makes and returns a action client goal
			\item \textbf{ensure-take-pose-goal-reached} \\
            This function makes sure the goal was reached
			\item \textbf{call-move-gripper-action} \\
			This function takes x,y,z coordiantes and four quaternion values to move the gripper to that position. 
		\end{itemize}
		\subsection{knowledge-client.lisp}
		The knowledge-client is an interface to knowledge, which is responsible for the request to the knowledgebase. For every request exists a function, which calls the prolog-query and returns the result. Implemented requests are:
			\begin{itemize}
				\item \textbf{knowledge-set-tables-source} \\ Sets all tables as source.
				\item \textbf{knowledge-set-ground-source} \\ Sets the ground as source.
				\item \textbf{knowledge-set-target-surfaces} \\ Sets all shelves as target surfaces.
			  \item \textbf{prolog-add-test-objects} \\ Adds test objects on the tables.
				\item \textbf{prolog-table-objects} \\ Returns all objects on the table.
				\item \textbf{prolog-object-goal} \\ Returns the goal shelf for an object.
			  \item \textbf{prolog-object-goal-pose} \\ Returns the goal pose for an object.
				\item \textbf{prolog-all-objects-in-shelf} \\ Returns all objects in the shelf.
				\item \textbf{prolog-next-object} \\ Returns the next object to grasp choosen by knowledge.
		    \item \textbf{prolog-object-dimensions} \\ Returns the dimension for an object (depth, width, height).
				\item \textbf{prolog-object-pose} \\ Returns the pose of an object as list.
				\item \textbf{prolog-table-pose} \\ Returns the pose of the table as list (x, y, z).
				\item \textbf{prolog-shelf-pose} \\ Returns the pose of the shelf as list (x, y, z).
				\item \textbf{prolog-object-in-gripper} \\ Returns the dimension of the object in gripper as list (depth, width, height).
			\end{itemize} 
		\subsection{knowledge-insertion-client.lisp}
		\subsection{navigation-action.lisp}
		\subsection{nlp-subscriber.lisp}
		The \textbf{nlp-subscriber} provides functions subscribing to the topics from nlp. At the topic \textit{suturo\_speech\_recognition\/hard\_commands} the commands "start", "stop" and "continue" are published, if they are recognised by nlp. The subscriber for the hard-commands \textit{static-command-listener} calls the function \textit{set-state-fluent} with the message as argument and a fluent-variable is set according to the command. The message for dynamic-commands is not defined yet.
		\subsection{text-to-speech.lisp}
		\subsection{obstacle-map-subscriber.lisp}
	    \begin{itemize}
	    	\item \textbf{obstacle-map-listener} \\
	    	This function subscribes the obstacle map and stores the current map in a variable.
	    	\item \textbf{save-obstacle-map} \\
	    	Makes a copy of the given message and stores it in a variable.
	    	\item \textbf{get-map-point} \\
	    	Returns the occupancy from a given position in the map.
	    \end{itemize}
		\subsection{poi-client.lisp}
 		\begin{itemize}
	    	\item \textbf{closest-poi} \\
	    	This function returns the next point of interest in relation to the passed point.
	    	\item \textbf{sorted-poi-by-distance} \\
	    	This function returns all points of interest sorted in relation to the given point.
	    	\item \textbf{point-listener} \\
	    	This function subscribes the point of interessts and stores them in a variable.
	    \end{itemize}
		\subsection{robosherlock-client-object.lisp}
		This file is responsible for the low level communication with the robosherlock pipeline which is responsible for objects.
		\begin{itemize}
			\item \textbf{init-robosherlock-object-action-client} \\
			This function initiates the connection to the robosherlock object action server
			\item \textbf{get-robosherlock-client} \\
			This function returns the robosherlock action client
			\item \textbf{make-action-goal} \\
			This function makes and returns a action client goal
			\item \textbf{call-robosherlock-object-pipeline} \\
			This function takes a array of regions and a boolean value for visualization, this function calls the robosherlock object pipeline with the input values and returns the result
		\end{itemize}
		\subsection{robosherlock-client-plane.lisp}
		This file is responsible for the low level communication with the robosherlock pipeline which is responsible for planes e.g. doors.
		\begin{itemize}
			\item \textbf{init-robosherlock-plane-action-client} \\
			This function initiates the connection to the robosherlock plane action server
			\item \textbf{get-robosherlock-client} \\
			This function returns the robosherlock action client
			\item \textbf{make-action-goal} \\
			This function makes and returns a action client goal
			\item \textbf{call-robosherlock-plane-pipeline} \\
			This function takes a array of regions and a boolean value for visualization, this function calls the robosherlock object pipeline with the input values and returns the result
		\end{itemize}
	\endgroup

\end{document}
