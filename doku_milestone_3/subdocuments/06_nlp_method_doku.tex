\documentclass[main.tex]{subfiles}

\begin{document}

	\begingroup

	\renewcommand{\cleardoublepage}{}

	\renewcommand{\clearpage}{}

	\chapter{NLP Function Documentation}
		\chapterauthor{Merete Bommarius}
		
	NLP has worked different sections of the field, starting from the beginning in the first Milestone. The group first made a list of potential commands the robot could receive in the previously mentioned challenges and then annotated them by hand to create a randomizer. This will be explained further in the following sections. This randomizer was then used to create thousands of commands which could be used to train the parser. So, it was NLP‘s job to train the parser so that the robot is able to understand what exactly it is that you want it to do, when you issue a more complex command involving one or more objects to act upon. 
For the second Milestone, the attention of the group shifted from natural language processing to natural language generation. The shift in attention was caused by the completion of one section and moving on to a different one. 
After the holidays, the project group as a whole got together and defined new goals for the second Milestone. They asked for hard coded commands like start and stop that would give the robot the signal to either start or stop a task or an action. Now, this technically counts as nlp, since the robot is processing language instead of generating it, but it was merely a side project. The main focus was on generating answers for the robot to give in certain situations, which will be elaborated later on.
		
	 \section{Natural Language Processing}
	 	\chapterauthor{Merete Bommarius}
	\endgroup

\end{document}
