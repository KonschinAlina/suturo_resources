\documentclass[main.tex]{subfiles}
\begin{document}
	\chapter{Introduction}
	\chapterauthor{}
	
	The goal of the Suturo project was the successful participation in the RoboCup@Home with the HSR from Toyota. During the RoboCup the HSR has to autonomously perform different task. 
	This year storing groceries and clean up have been the task we focused on. During the the storing groceries task the HSR has to identify objects on a table and store reasonably in a shelf. In the clean up task several objects are distributed in a room and have to be brought to a bin.
	
	\subsection{Challenges} 
	These task result in several challenges:
	\begin{itemize}
		\item Moving the robot. In order to get from one position to another for example the table and shelf we need to be able to move the robot. 
		\item Localizing the robot: The robot needs to know where it is in the world.
		\item Finding an object
		\item Recognizing an object
		\item Determine the position and dimensions of an object
		\item Grasping an object: This for a human seemingly easy task is quite complex for an actual robot since several factors have to be considered. First of all the kinematic chain of the robot. This is an description of how the robot can move. Now in order to grasp an Object It has to be calculated how these joints are moved. Additionally the environment has to be taken into account to avoid any collisions ...
		\item ...
	\end{itemize}
	
	\textit{(Further details to the problems should be given)}	
	

\end{document}
