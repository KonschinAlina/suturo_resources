\documentclass[main.tex]{subfiles}
\begin{document}
	
	\chapter{Challenges}
	\chapterauthor{Fabian Rosenstock}

\section{General}

While we were working on this milestones goals, we had to change our way of working completely, because of the COVID-19 outbreak.
We could not work with the robot anymore and the RoboCup got canceled. This lead to a huge loss of motivation, since we lost our main goal and could not personally work together anymore.
The situation forced us to work in a simulation. We had to set up a fitting simulation environment and find a way to properly integrate and test our system from home. This did cost a lot of time, where we were not able to properly work on the project.
These changes had some positive side effects. Since we worked from home, it was easier to find someone who was able to help, when we encountered some problem. Most of the time we worked on one computer, which lead to an better overview over the work and the problems of other groups.

\section{Navigation}
At the end of the last milestone it became apparent, that the HSR hat some serious problems moving around. The origin of this problem most likely lay in the actual hardware. Due to Covid-19 access to the HSR and further investigations became impossible. The navigation inside of the simulation worked flawlessly which led to a drop of the problem. Instead the focus was shifted to the object finder. A small node trying to find objects on the ground with the laser scanner. The program still had some problems correctly detecting the position of an object while the HSR moved around. Additionally it had been tweaked to work with the original HSR, which made an adjustment of its parameters necessary.  

\section{Manipulation}

The manipulation group worked a lot with the robot, which worked well, since the communication and teamwork was good.
Because of that they were strongly affected by the changes made due to COVID-19. After the work with the robot was not possible anymore a lot of motivation was lost.

\section{Knowledge}

A clear division of tasks lead to good teamwork and easier communication in the knowledge group.
The changes made due to the COVID-19 outbreak, did not affect the knowledge group as much as the other groups, since they did not have to work as closely with the robot.

\section{Planning}

The COVID-19 changes made the work for planning really hard, since they worked close to the robot. Because of that they could not really work while the changes were made and had to adapt to the new situation, which took some time.
This lead to a loss in motivation. Despite that the group work did function well, with good communication and clear division of tasks.

\section{Perception}

The teamwork and communication worked well, despite COVID-19. Although the changing working and environmental conditions lead to a loss of motivation.


	
\end{document}
