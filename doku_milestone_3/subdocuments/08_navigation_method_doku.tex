\documentclass[main.tex]{subfiles}

\begin{document}

	\begingroup

	\renewcommand{\cleardoublepage}{}

	\renewcommand{\clearpage}{}

	\chapter{Navigation Function Documentation}

		\chapterauthor{Marc Stelter}
		
		\section{mapping\_hector\_slam}
		Contains a simple launch file starting hector\_slam with the correct parameters for the hsr.
		
		\section{obstacle\_finder.py}
		Uses the laser scanner in accordance with the map to find objects.
		
		\begin{itemize}
			\item Input: 
				\subitem nav\_msgs/OccupancyGrid
				\subitem sensor\_msgs/LaserScan
			\item Parameter:
				\subitem occ\_threshold 
				Minimum value for a cell in the OccupancyGrid to be occupied
				\subitem min\_scans\_cluster
				Minimum number of laser rays to form a cluster
				\subitem min\_percentage\_covered
				Percentage of a cluster covered by the map to be part of it
				\subitem rad\_neighbour (m)
				Max distance between cells of a cluster
				\subitem error\_map (m)
				Allowed deviation in the map
				\subitem use\_every\_n\_scan
				Skip ecery nth LaserScan message 
			\item  Publish:
				\subitem Topic: /object\_finder
				\subitem Type: geometry\_msgsPoseArray
		\end{itemize}
		
		\section{nav\_fix.py}
		Small node that takes in the original navigation goal and publishes an intermediate goal upon failure. The original goal gets published 3 times.
		
		This small node has proven to be useful inside the hsr lab, since some cables seem to be triggering the magnetic sensors.

	\endgroup

\end{document}
