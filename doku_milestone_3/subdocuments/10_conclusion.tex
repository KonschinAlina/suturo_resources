\documentclass[main.tex]{subfiles}
\begin{document}
	
	\chapter{Conclusion}
	\chapterauthor{Jan-Frederik Stock}
	Due to the situation caused by Covid-19, a drastic change in the way the SUTURO project worked became necessary.  The project was forced to stop working on the actual robot, and moved the integration into a simulation. This process was a very time consuming task in the beginning, which hindered the work on new features.
	
	For many members of the project, the situation caused loss of motivation, because the goal of the project, the RoboCup was cancelled and working on an actual robot was no longer possible. 
	
	A general observation was that the workload balanced in comparison to the previous milestones, people who had had a very high workload now worked less, while people who had previously not worked as hard had a higher workload in this milestone.
	
	In the last demonstration, both the clean up task and the storing groceries task were shown, with the clean up task working flawlessly and the storing groceries getting stuck while placing the first object into the shelf.
	
	In conclusion, the SUTURO project group was able to deliver integrated solutions for the planned roboCup tasks and add additional features in the third milestone in spite of the obstacles posed by the Corona virus outbreak. 


	
\end{document}
